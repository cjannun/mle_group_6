\documentclass[10pt]{article}

\usepackage[left=0.8in,right=0.8in,top=0.15in,bottom=0.8in]{geometry}
\usepackage{xcolor}
\usepackage{hyperref}

\hypersetup{colorlinks=true,linkcolor=blue,urlcolor=blue}
\urlstyle{rm}
\usepackage{url}

\title{CAI 4104/6108: Machine Learning Engineering\\
	\Large Project Proposal: {\textcolor{purple}{Title of the Project}}} %% TODO: replace with the title of your project

%% TODO: your name and email go here (all members of the group)
%% Add/remove as needed and designate a point of contact
\author{
        C.J. Annunziato \\{\em (Point of Contact)} \\
        c.annunziato@ufl.edu\\
        \and
        Aaron Shumer \\
        shumera@ufl.edu\\
        \and
        Group Member 3's Name \\
        email3@ufl.edu\\
        \and
        Group Member 4's Name \\
        email4@ufl.edu\\
        \and
        Group Member 5's Name \\
        email5@ufl.edu\\
}

% set the date to today
\date{\today}


\begin{document} % start document tag

\maketitle


%%% Remember: writing counts! (try to be clear and concise.)
%%% the whole proposal should be about 2 pages (in 11pt font)


%% TODO: briefly describe your proposed task and data
%% Must address:
%% - What is the task?
%% - What is the dataset? How large is it? Include a link
%%
\section*{Task \& Dataset}


\paragraph{Task.} 
% TODO: What is the task about? Be specific. A few sentences is fine. 

\begin{itemize}

\item {\bf Description: } Write here. % TODO: A few sentences describing the task. What is the task about?

\item {\bf Type: } Write here. % TODO: What is the type of task?   (Is it supervised/unsupervised, it is classification/regression?) If it is classification, is it multi-class? Be as a specific as possible.

\end{itemize}

\paragraph{Dataset.} 
%
\begin{itemize}

\item {\bf Description: } Write here. % TODO: A few sentences describing the dataset (what kind of data is it? where does it come from?). Note: no toy datasets please!

\item {\bf Size: } Write here. % TODO: How many data points/instances does it contain? How many features? What are the features (how are they encoded)?

\item {\bf Available at: } Write URL here. Example: \url{https://www.url\_of\_data.com/} % TODO: Where is the dataset? Make sure you include a link and explain any access restrictions.

\end{itemize}

% TODO: remove the following example of citation and footnote before you submit.
If necessary you can cite scholarly work like this~\cite{murphy2022probabilistic}. You can also include URLs in a footnote like this.\footnote{UCI repository: \url{https://archive.ics.uci.edu/}.}


%% TODO: write about your evaluation methodology using the provided template.
%%
\section*{Evaluation}

\paragraph{Methodology.} 
% TODO: What methodology will you use to evaluate your ML pipeline? Are you going to follow best practices? How are you going to split the data (train-test)? A few sentences is fine. 

\begin{itemize}

\item {\bf Description: } Write here. % TODO: A few sentences describing the methodology.

\item {\bf CV/Split: } Write here. % TODO: How are you going to split the data? Are you using an existing split or creating your own? Will you do cross-validation? Be as a specific as possible.

\end{itemize}


\paragraph{Metrics.} 
% TODO: What metrics will you use to evaluate your ML pipeline? Give at least two and explain how you will compute them. A few sentences is fine. 

\begin{itemize}

\item {\bf Metric 1: } Write here. % TODO: A sentence or two naming & describing the metric and a formula (if not trivial).

\item {\bf Metric 2: } Write here. % TODO: A sentence or two naming & describing the metric and a formula (if not trivial).

\end{itemize}



\paragraph{Baselines.} 
% TODO: What baselines or point of comparison will you use to evaluate your ML pipeline? Give at least two and provide a reference for them. Important: only one can be a "simple" baseline (e.g., random guessing or mean model). The other must be a citable proper scientific or engineering work.

\begin{itemize}

\item {\bf Baseline 1: } Write here. % TODO: A few sentences and a (possibly) a link or citation. If known/applicable, give the performance of the baseline according to your chosen metrics.

\item {\bf Baseline 1: } Write here. % TODO: A few sentences and a (possibly) a link or citation. If known/applicable, give the performance of the baseline according to your chosen metrics.

\end{itemize}

% TODO: remove the following example sentence before you submit.
Here you must provide citations and/or links.



% references here
{\small
\bibliography{refs}
\bibliographystyle{abbrv}
}

\end{document} % end tag of the document
